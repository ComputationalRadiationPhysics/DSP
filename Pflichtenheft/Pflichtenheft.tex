\documentclass[paper=a4, headings=small]{scrartcl}

\usepackage[T1]{fontenc}
\usepackage[utf8]{inputenc}
\usepackage[ngerman]{babel}
\usepackage{amsmath}
\usepackage{pbox}
\usepackage{tabu}
\usepackage{hyperref}
\usepackage{color}
\usepackage{listings}

%Code for Code listings copied from http://tex.stackexchange.com/questions/81834/whats-the-best-way-to-typeset-c-codes-in-latex
\definecolor{listinggray}{gray}{0.9}
\definecolor{lbcolor}{rgb}{0.9,0.9,0.9}
\lstset{
backgroundcolor=\color{lbcolor},
    tabsize=4,    
%   rulecolor=,
    language=[GNU]C++,
        basicstyle=\scriptsize,
        upquote=true,
        aboveskip={1.5\baselineskip},
        columns=fixed,
        showstringspaces=false,
        extendedchars=false,
        breaklines=true,
        prebreak = \raisebox{0ex}[0ex][0ex]{\ensuremath{\hookleftarrow}},
        frame=single,
        numbers=left,
        showtabs=false,
        showspaces=false,
        showstringspaces=false,
        identifierstyle=\ttfamily,
        keywordstyle=\color[rgb]{0,0,1},
        commentstyle=\color[rgb]{0.026,0.112,0.095},
        stringstyle=\color[rgb]{0.627,0.126,0.941},
        numberstyle=\color[rgb]{0.205, 0.142, 0.73},
%        \lstdefinestyle{C++}{language=C++,style=numbers}’.
}
\lstset{
    backgroundcolor=\color{lbcolor},
    tabsize=4,
  language=C++,
  captionpos=b,
  tabsize=3,
  frame=lines,
  numbers=left,
  numberstyle=\tiny,
  numbersep=5pt,
  breaklines=true,
  showstringspaces=false,
  basicstyle=\footnotesize,
%  identifierstyle=\color{magenta},
  keywordstyle=\color[rgb]{0,0,1},
  commentstyle=\color{Darkgreen},
  stringstyle=\color{red}
 }
  
  
\begin{document}

\begin{titlepage}
\begin{center}
	\large{Pflichtenheft\\ Helmholtz-Zentrum Dresden-Rossendorf } \\[10mm]
	\textbf{\Large{Digital Signal Processing}}\\[5mm]
	\rule{\linewidth}{0.5mm}
  \textbf{\LARGE{ --- Beschreibung hier eingügen ---}}
	\rule{\linewidth}{0.5mm}\\[3cm]

\parbox{0cm}{
\begin{tabbing}
  Fabian Jung	              \hspace{2cm}	\= Nico Wehmeyer  \hspace{2cm}	\= Richard Pfeifer\\
  Mat.Nr.:  3755341                         \> Mat.Nr.: 3658043			    \> Mat.Nr.: ??????\\
  Diplom Informatik 			            \> Diplom Informatik     	    \> Diplom Physiker?\\
\end{tabbing}
}

\vfill
{\large \today}
\end{center}

\end{titlepage}

\tableofcontents
\newpage
\input{Levenberg_Marquardt.tex}
\newpage
\input{Speichertransfer.tex}
\newpage
\input{Host.tex}

\end{document}
