\section{Optimierungsalgorithmus}
	\begin{itemize}
		\item Aufgabe
		
		Der Algorithmus soll die gemessenen Daten näherungsweise als Funktion interpretieren. (Die Daten sind gleichmäßig über die Zeit verteilt und repräsentieren die Funktionswerte.)\\
		Anschließend sollen der Anfangs-, End- und Maximalwert der Funktion bestimmt werden.\\
		
		\item Implementierungsschritte
		
		\begin{itemize}
			\item Levenberg-Marquardt-Algorithmus
			\item double zu short int konvertieren (betrifft Messdaten)
			\item kernel-Methode (statt main)
			\item Speicher der Grafikkarte nutzen (statt malloc)
			\item Cuda Array nutzen (statt Array)
			\item Berechnung von Anfangs- und Endfunktionswerten (ggf. mitteln)
			\item Rückgabe der Ergebnisdaten
			\item ggf. kernel zu sub-kernel parallelisieren
			\item ggf. andere Fit-Funktion (z. B. $e^{-x^2}$ mit entsprechenden Parametern) verwenden oder Daten vor Bestimmung der quadratischen Funktion trimmen
		\end{itemize}
	\end{itemize}
